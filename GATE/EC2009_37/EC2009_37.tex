\documentclass[12pt,a4paper]{article}
\usepackage{geometry}
\geometry{margin=1in}
\usepackage{caption}
\usepackage{float}
\usepackage{amsmath}
\usepackage{enumitem}
\usepackage{graphicx}
\usepackage{multicol}

\title{\textbf{IMPLEMENTATION OF NAND NOR LATCH USING ARDUINO AND LEDS}}

\author{
\begin{tabular}{c}
\hspace{-5cm}\textbf{Ch.Pranai} \\
\hspace{-5cm}\texttt{pranai.fwc1@iiitb.ac.in} \\
\textbf{\hspace{-2cm}COMETFWC022  IITB Future Wireless Communication (FWC) ASSIGNMENT}
\end{tabular}
}

\date{July 06, 2025}

\begin{document}
%\includegraphics[width=\linewidth]{/storage/emulated/0/test/IMG-20250515-WA002}
\maketitle

% ---------- First Page ----------

\begin{figure}[H]
  \centering
  \begin{minipage}[t]{0.48\textwidth}
    \section*{Question}
    \small
    \noindent\textbf{Q.38} \quad Refer to the NAND and NOR latches shown in the figure. The inputs ($P_1$, $P_2$) for both the latches are first made (0,1) and then, after a few seconds, made (1,1). The corresponding stable outputs ($Q_1$, $Q_2$) are observed.

    \vspace{0.3cm}
    \includegraphics[width=\linewidth]{/storage/emulated/0/test/IMG_20250708_150815.png} 

    \vspace{0.5cm}
    \section*{Truth Table for NAND and NOR Latch}
    \renewcommand{\arraystretch}{1.3}
    \begin{tabular}{|c|c|c|c|c|c|}
      \hline
      \textbf{Latch Type} & \textbf{$P_1$} & \textbf{$P_2$} & \textbf{$Q_1$} & \textbf{$Q_2$} & \textbf{State} \\
      \hline
      NAND & 0 & 1 & 1 & 0 & Set \\
      NAND & 1 & 1 & 1 & 0 & Hold (Set) \\
      \hline
      NOR & 0 & 1 & 1 & 0 & Reset \\
      NOR & 1 & 1 & 0 & 0 & Invalid \\
      \hline
    \end{tabular}
  \end{minipage}\hfill
  \begin{minipage}[t]{0.48\textwidth}
    \section*{Components}
    \small
    \renewcommand{\arraystretch}{1.3}
    \begin{tabular}{|l|l|l|}
      \hline
      \textbf{Component} & \textbf{Value} & \textbf{Quantity} \\ \hline
      Arduino Board & -- & 1 \\ \hline
      Jumper Wires & M-F & 10 \\ \hline
      Push Buttons & -- & 2 \\ \hline
      Breadboard & -- & 1 \\ \hline
      USB Cable & -- & 1 \\ \hline
      LED & -- & 2 \\ \hline
      Resistors & 220~$\Omega$, 10k~$\Omega$ & 2, 2 \\ \hline
    \end{tabular}

    \vspace{0.8cm}
    \section*{Setup}
    \small
    \begin{enumerate}[left=0pt]
      \item Connect push button P1 to digital pin D2 with a 10k$\Omega$ pull-down resistor.
      \item Connect push button P2 to digital pin D3 with a 10k$\Omega$ pull-down resistor.
      \item Connect LED Q1 to digital pin D12 through a 220$\Omega$ resistor to ground.
      \item Connect LED Q2 to digital pin D13 through a 220$\Omega$ resistor to ground.
      \item Upload latch emulation code to Arduino to simulate NAND latch behavior using inputs P1 and P2.
    \end{enumerate}
  \end{minipage}
\end{figure}

% ---------- Second Page with Left-aligned Implementation ----------

\newpage

\begin{figure}[H]
  \centering
  \begin{minipage}[t]{0.48\textwidth}
    \section*{Implementation}
    \small
    \begin{enumerate}[left=0pt]
      \item Define digital pins D2 and D3 as inputs for buttons P1 and P2.
      \item Define digital pins D12 and D13 as outputs for LEDs Q1 and Q2.
      \item Use \texttt{pinMode()} in \texttt{setup()} to configure input and output pins.
      \item In \texttt{loop()}, read inputs P1 and P2 using \texttt{digitalRead()} and apply NAND or NOR latch logic.
      \item Use \texttt{digitalWrite()} to display Q1 and Q2 states on LEDs according to the selected latch type.
    \end{enumerate}
  \end{minipage}\hfill
  \begin{minipage}[t]{0.48\textwidth}
    % Right side left intentionally blank or for future use
  \end{minipage}
\end{figure}

\end{document}
