\documentclass[12pt,a4paper]{article}
\usepackage{geometry}
\geometry{margin=1in}
\usepackage{caption}
\usepackage{float}
\usepackage{amsmath}

\title{\textbf{IMPLEMENTATION OF KMAP BOOLEAN LOGIC WITH ARDUINO}}
\author{Ch.Pranai \\ \texttt{pranai.fwc1@iiitb.ac.in} \\
 \hspace{-0.3cm}COMETFWC022\hspace{0.5cm} IITB Future Wireless Communication (FWC)\hspace{0.4cm} ASSIGNMENT}
\date{July 06, 2025}

\begin{document}

\maketitle

\begin{figure}[H]
  \centering
  \begin{minipage}[t]{0.48\textwidth}
    \section*{Abstract}
    \small
    The minimal sum-of-products expression for the logic function $f$ represented by the given Karnaugh map is:

    \vspace{0.3cm}

    \renewcommand{\arraystretch}{1.3}
    \begin{tabular}{c|c|c|c|c}
      PQ/RS & 00 & 01 & 11 & 10 \\ \hline
      00 & 0 & 1 & 0 & 0 \\ \hline
      01 & 0 & 1 & 1 & 1 \\ \hline
      11 & 1 & 1 & 1 & 0 \\ \hline
      10 & 0 & 0 & 1 & 0 \\
    \end{tabular}

    \vspace{0.3cm}

    Options:
    \begin{enumerate}
      \item[(A)] QS + $\overline{P}R\overline{S}$ + PQR + $\overline{P}RS$ + P$\overline{Q}R$
      \item[(B)] QS + $\overline{P}R\overline{S}$ + PQR + $\overline{P}RS$
      \item[(C)] $\overline{P}RS$ + PQR + $\overline{P}RS$ + P$\overline{Q}R$
      \item[(D)] $\overline{P}RS$ + PQR + P$\overline{Q}R$
    \end{enumerate}

    \vspace{0.5cm}

    \section*{Kmap}

   \renewcommand{\arraystretch}{1.3}
    \begin{tabular}{c|c|c|c|c}
      PQ/RS & 00 & 01 & 11 & 10 \\ \hline
      00 & 0 & 1 & 0 & 0 \\ \hline
      01 & 0 & 1 & 1 & 1 \\ \hline
      11 & 1 & 1 & 1 & 0 \\ \hline
      10 & 0 & 0 & 1 & 0 \\
    \end{tabular}

    \vspace{0.3cm}
  \end{minipage}\hfill
  \begin{minipage}[t]{0.48\textwidth}
    \section*{Components}
    \small
    \renewcommand{\arraystretch}{1.3}
    \begin{tabular}{|l|l|l|}
      \hline
      \textbf{Component} & \textbf{Value} & \textbf{Quantity} \\ \hline
      Arduino Board & -- & 1 \\ \hline
      Jumper Wires & M-F & 10 \\ \hline
      Push Buttons & -- & 4 \\ \hline
      Breadboard & -- & 1 \\ \hline
      USB Cable & -- & 1 \\ \hline
      LED & -- & 1 \\ \hline
      Resistors & 220~$\Omega$, 10k~$\Omega$ & 5 \\ \hline
    \end{tabular}

    \vspace{0.8cm}

    \section*{Setup}
    \small
    \begin{enumerate}
      \item Connect buttons to pins D2–D5 for inputs $P$, $Q$, $R$, $S$, each with a 10k$\Omega$ pull-down resistor to ground.
      \item Connect an LED to pin D13 through a 220~$\Omega$ resistor to ground.
      \item Write Arduino code to read inputs and compute $f = {!}P \cdot R \cdot S + P \cdot Q \cdot R + {!}P \cdot R \cdot {!}S + {!}P \cdot Q \cdot R$.
      \item Use \texttt{digitalWrite(13, f)} to control the LED based on the computed output.
      \item Upload the code and test by pressing buttons to check when the LED turns ON for $f = 1$.
    \end{enumerate}
  \end{minipage}
\end{figure}

\newpage

\begin{figure}[H]
  \centering
  \begin{minipage}[t]{0.48\textwidth}
    \section*{Implementation}
    \small
    Connect push buttons to pins D2–D5 for inputs $P$, $Q$, $R$, $S$ with 10k$\Omega$ pull-down resistors. \\
    
    Connect an LED to pin D13 through a 220$\Omega$ resistor to ground. \\
    
    Write Arduino code to read inputs and compute $f = {!}P \cdot R \cdot S + P \cdot Q \cdot R + {!}P \cdot R \cdot {!}S + {!}P \cdot Q \cdot R$. \\
    
    Use \texttt{digitalWrite(13, f)} to control the LED based on the computed output. \\
    
    Upload the code and test by pressing buttons to check when the LED turns ON for $f = 1$.
  \end{minipage}
  \hfill
  \begin{minipage}[t]{0.48\textwidth}
    % Right side intentionally left blank for white space
  \end{minipage}
\end{figure}

\end{document}
