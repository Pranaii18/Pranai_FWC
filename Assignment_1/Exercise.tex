\documentclass[12pt]{article}
\usepackage{amsmath, amssymb}
\usepackage{geometry}
\usepackage{array}
\usepackage{xcolor}
\usepackage{fancyhdr}
\usepackage{multicol}
\usepackage{enumitem}
\usepackage{booktabs}
\usepackage{fancybox}
\geometry{margin=1in}
\setlength{\parindent}{0pt}
\setlength{\parskip}{6pt}
\renewcommand{\arraystretch}{1.5}
\usepackage{graphicx}

\begin{document}

\pagestyle{empty} % Start with empty page style

\thispagestyle{fancy} % Apply fancy style only to the first page
\fancyhf{} % Clear header and footer
\renewcommand{\headrulewidth}{0pt} % Remove header rule

\fancyhead[L]{% Left header
	\includegraphics[width=8cm, height=1.7cm]{1.png} % Adjust dimensions
}
\fancyhead[R]{% Right header
    Name: PRANAI \\
    Batch: COMETFWC23\\
    Date: 19 MAY 2025 
}
\noindent\rule{\textwidth}{0.5pt}


\vspace{-0.5em}
\section*{\textcolor{blue}{\large EXERCISE 5.1}}
\begin{enumerate}
\item In which of the following situations, does the list of numbers involved make an arithmetic progression, and why?

\begin{enumerate}[label=(\alph*)]
    \item The taxi fare after each km when the fare is 15 for the first km and 8 for each additional km.
    \item The amount of air present in a cylinder when a vacuum pump removes $\frac{1}{4}$ of the air remaining in the cylinder at a time.
    \item The cost of digging a well after every metre of digging, when it costs 150 for the first metre and rises by 50 for each subsequent metre.
    item The amount of money in the account every year, when 1000 is deposited at compound interest at $8\%$ per annum.
\end{enumerate}

\item Write first four terms of the AP, when the first term \( a \) and the common difference \( d \) are given:

\begin{multicols}{2}
\begin{enumerate}[label=(\alph*)]
    \item \( a = 10,\ d = 10 \)
    \item \( a = -10,\ d = 0 \)
    \item \( a = 4,\ d = -3 \)
    \item \( a = -1,\ d = \frac{1}{2} \)
    \item \( a = 1.25,\ d = -0.25 \)
\end{enumerate}
\end{multicols}

\item For the following APs, write the first term and the common difference:

\begin{multicols}{2}
\begin{enumerate}[label=(\alph*)]
    \item \( 3,\ 1,\ -1,\ -3,\ \ldots \)
    \item \( -5,\ -1,\ 3,\ 7,\ \ldots \)
    \item \( \frac{1}{3},\ \frac{5}{3},\ 3,\ \frac{13}{3},\ \ldots \)
    \item \( 0.6,\ 1.7,\ 2.8,\ 3.9,\ \ldots \)
\end{enumerate}
\end{multicols}

\item Which of the following are APs? If they form an AP, find the common difference \( d \) and write three more terms:

\begin{multicols}{2}
\begin{enumerate}[label=(\alph*)]
    \item \( 2,\ 4,\ 8,\ 16,\ \ldots \)
    \item \( 5,\ 3,\ 1,\ -1,\ \ldots \)
    \item \( 1,\ -1.2,\ -3.2,\ -5.2,\ -7.2,\ \ldots \)
    \item \( \sqrt{3},\ \sqrt{8},\ \sqrt{18},\ 3 + \sqrt{2},\ \ldots \)
    \item \( 1,\ 3,\ 5,\ 7,\ \ldots \)
    \item \( 0,\ -4,\ -8,\ -12,\ \ldots \)
    \item \( 0.2,\ 0.22,\ 0.222,\ 0.2222,\ \ldots \)
    \item \( \frac{1}{2},\ \frac{1}{2},\ \frac{1}{2},\ \ldots \)
    \item \( 1,\ 3,\ 9,\ 27,\ \ldots \)
    \item \( a,\ 2a,\ 3a,\ 4a,\ \ldots \)
    \item \( a,\ a^2,\ a^3,\ a^4,\ \ldots \)
    \item \( \sqrt{2},\ \sqrt{8},\ \sqrt{18},\ \ldots \)
    \item \( \sqrt{3},\ \sqrt{6},\ \sqrt{9},\ \sqrt{12},\ \ldots \)
    \item \( 1^2,\ 3^2,\ 5^2,\ 7^2,\ \ldots \)
\end{enumerate}
\end{multicols}
\end{enumerate}

\vspace{0.5em}
\section*{\textcolor{blue}{\large EXERCISE 5.2}}
\begin{enumerate}
\item Fill in the blanks in the following table, given that \( a \) is the first term, \( d \) the common difference and \( a_n \) the \( n \)th term of the AP:

\[
\begin{tabular}{|c|c|c|c|c|}
\hline
\textbf{(i)} & 7 & 3 & 8 & \_\_ \\
\hline
\textbf{(ii)} & \_\_ & \_\_ & 10 & 0 \\
\hline
\textbf{(iii)} & -18 & \_\_ & \_\_ & -10 \\
\hline
\textbf{(iv)} & -18.9 & 2.5 & 105 & 3.6 \\
\hline
\textbf{(v)} & 3.5 & 0 & 105 & \_\_ \\
\hline
\end{tabular}
\]

\item Choose the correct option and justify:

\begin{enumerate}[label=(\alph*)]
    \item 30th term of the AP: \( 10,\ 7,\ 4,\ \ldots \) is\\
    (A) 97 \quad (B) 77 \quad (C) -77 \quad (D) -87

    \item 11th term of the AP: \( -3,\ -\frac{1}{2},\ 2,\ \ldots \) is\\
    (A) 28 \quad (B) 22 \quad (C) -38 \quad (D) $-48\dfrac{1}{2}$
\end{enumerate}



\item In the following APs, find the missing terms in the boxes:

\begin{enumerate}[label=(\alph*)]
    \item \( 2,\ \fbox{\rule{0pt}{1em}\hspace{1em}},\ \fbox{\rule{0pt}{1em}\hspace{1em}},\ 26 \)
    \item \( \fbox{\rule{0pt}{1em}\hspace{1em}},\ 13,\ \fbox{\rule{0pt}{1em}\hspace{1em}},\ 3 \)
    \item \( \fbox{\rule{0pt}{1em}\hspace{1em}},\ \fbox{\rule{0pt}{1em}\hspace{1em}},\ 9,\ \frac{1}{2} \)
    \item \( \fbox{\rule{0pt}{1em}\hspace{1em}},\ \fbox{\rule{0pt}{1em}\hspace{1em}},\ \fbox{\rule{0pt}{1em}\hspace{1em}},\ -10 \)
    \item \( 2,\ \fbox{\rule{0pt}{1em}\hspace{1em}},\ \fbox{\rule{0pt}{1em}\hspace{1em}},\ \fbox{\rule{0pt}{1em}\hspace{1em}},\ -10 \)
    \item \( \fbox{\rule{0pt}{1em}\hspace{1em}},\ 38,\ \fbox{\rule{0pt}{1em}\hspace{1em}},\ \fbox{\rule{0pt}{1em}\hspace{1em}},\ \fbox{\rule{0pt}{1em}\hspace{1em}},\ -22 \)
\end{enumerate}


\item Which term of the AP: \( 3,\ 8,\ 13,\ 18,\ \ldots \) is 78?

\item Find the number of terms in each of the following APs:

\begin{enumerate}[label=(\alph*)]
    \item \( 7,\ 13,\ 19,\ \ldots,\ 205 \)
    \item \( 18,\ 15,\ 12,\ \ldots,\ -47 \)
\end{enumerate}


\item Check whether $-150$ is a term of the AP: $11,\ 8,\ 5,\ 2,\ \ldots$

\item\ Find the 31st term of an AP whose 11th term is 38 and the 16th term is 73.

\item An AP consists of 50 terms of which 3rd term is 12 and the last term is 106. Find the 29th term.

\item If the 3rd and the 9th terms of an AP are 4 and -8 respectively, which term of this AP is 0?

\item  The 17th term of an AP exceeds its 10th term by 7. Find the common difference.

\item Which term of the AP: \( 3,\ 15,\ 27,\ 39,\ \ldots \) will be 132 more than its 54th term?




\item Two APs have the same common difference. The difference between their 100th terms is 100. What is the difference between their 1000th terms?
\item How many three-digit numbers are divisible by 7?


\item How many multiples of 4 lie between 10 and 250?

\item For what value of \( n \), are the \( n \)th terms of two APs: \( 63,\ 65,\ 67,\ \ldots \) and \( 3,\ 10,\ 17,\ \ldots \) equal?


\item  Determine the AP whose third term is 16 and the 7th term exceeds the 5th term by 12.

\end{enumerate}

\end{document}
